\clearpage
\null
\vfill
\begin{center}
	Ce document a été réalisé grâce à \XeLaTeX.
\end{center}
\newpage


%Copyright avec notice de copyright, infos sur édition, publication, données de catalogue, etc. Crédit pour design, production, edition et illustration?


\null
\vspace{.1\textheight}
\section*{Remerciements}
Je tiens à remercier pour leur aide précieuse durant la réalisation de ce travail:

\textbf{Mme Purro, }ma professeure accompagnante, qui aura accepté de me suivre malgré un sujet qui, au premier abord, lui paraissait complètement étranger.

\textbf{Ma famille, }pour son support inébranlable et ses conseils nombreux et pertinents --- notamment concernant les points sur lesquels j'aurais dû me concentrer, ce que je n'ai pas fait et ce qui n'était pas très malin.

\textbf{Mathilde, }pour m'avoir écouté débiter des monologues (sûrement interminables d'ailleurs) à propos de ce projet et pour son avis éclairé sur mes multiples problèmes.

\textbf{Les communautés web et les forums} sans lesquels ce travail n'aurait jamais été possible.

\vspace*{.5cm}
Je tiens également à remercier la \textbf{communauté \LaTeX\ }pour l'outil formidable qu'elle a réussi à créer, élément fondamental de la mise en page de ce document.

\newpage
\null
\vspace{.1\textheight}
\section*{Normes typographiques utilisées dans ce documet}
Les mots suivis d'un astérisque\definition\ sont des termes, souvent relatifs au domaine du jeu vidéo ou de l'informatique, qui peuvent ne pas faire partie du vocabulaire courant. Ils sont définis dans l'annexe \ref{chap:vocabulaire} \enquote{\nameref{chap:vocabulaire}} afin de lever toute incertitude.

\normalInfo{Blocs bleus}{Les paragraphes typographiés de cette façon indiquent les informations utiles.}

\vspace*{-\baselineskip}
\warningInfo{Blocs jaunes}{Cette mise en page indiquera les informations importantes.}

\vspace*{-\baselineskip}
\criticalInfoDarkRed{Blocs rouges}{Les informations ainsi présentées sont critiques, absolument nécessaires pour ce travail.}


\null
\vfill







