%\chapter*{Résumé}

%Copyright avec notice de copyright, infos sur édition, publication, données de catalogue, etc. Crédit pour design, production, edition et illustration?

\null
\vspace{.1\textheight}
\section*{Remerciements}
Je tiens à remercier pour leur aide précieuse durant la réalisation de ce travail:

\textbf{Mme Purro, }ma professeure accompagnate, qui aura accepté de me suivre malgré un sujet qui, du premier abord, lui paraissait complètement étranger.

\textbf{Ma famille, }pour leur support inébranlable et leurs conseils nombreux et pertinent -- notamment concernant les points sur lesquels j'aurais dû me concentrer, ce que je n'ai pas fait et ce qui n'était pas très malin.

\textbf{Mathilde, }pour m'avoir écouté débiter des monologues surement interminables à propos de ce projet et pour ton avis éclairé sur mes multiples problèmes.

\textbf{Les communautés web et forums} sans lequel ce travail n'aurait jamais été possible.




\newpage
\null
\vspace{.1\textheight}
\section*{Typographie de ce document}
Les mots suivis d'une étoile\definition\ sont des termes, souvent relatifs au domaine du jeu vidéo ou de l'informatique, qui peuvent ne pas faire partie du vocabulaire courant. Ils sont définis dans l'annexe \ref{chap:vocabulaire} \enquote{\nameref{chap:vocabulaire}} afin de lever toute incertitude.

\vspace*{\baselineskip}
\normalInfo{Blocks bleus}{Les paragraphes typographiés de cette façon indique les informations utiles.}

\vspace*{-\baselineskip}
\warningInfo{Blocs jaunes}{C'est cette mise en page qui indiquera les informations importantes.}

\vspace*{-\baselineskip}
\criticalInfo{Blocs rouges}{Les informations ainsi présentées sont critiques, absolument nécessaires pour ce travail.}

\vfill
\begin{center}
	Ce document a été entièrement réalisé \enquote{à la main} avec le système de typographie \XeLaTeX.
\end{center}

	