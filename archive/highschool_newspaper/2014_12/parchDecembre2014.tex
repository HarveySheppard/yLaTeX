\documentclass[a4paper, 11pt, twoside]{article}
\usepackage[utf8]{inputenc}
\usepackage[T1]{fontenc}
\usepackage[francais]{babel}

\usepackage[usenames, dvipsnames, svgnames, table]{xcolor} %nouvelles couleurs, couleurs personnalis�e
\usepackage{graphicx} %gestion des images
\usepackage{csquotes} % pour les citations (guillemets adapt�s � la langue, etc.), commande: \enquote{text}
\usepackage{url} %url
\usepackage{blindtext} %pour g�n�rer du texte
\usepackage{setspace} %gestion des interlignes
\usepackage{fancyhdr} %header personnalis�
\usepackage{multicol} %g�re les colonnes multiples (plus de deux)
\usepackage{wrapfig} %permet d'avoir des images avec du texte qui s'enroule autour
\usepackage{lettrine} %pour les capitales initiales
\usepackage{mathptmx} %police times
\usepackage{rotating} %pour pouvoir faire tourner du texte (90�, 180�, etc)
\usepackage[normalem]{ulem} %pour la gestion de tous les soulign�s
\usepackage[large]{cwpuzzle} %pour les sudokus
\usepackage{pas-crosswords} %pour les mots-crois�s
\usepackage{tcolorbox} %Pour les boites avec des angles arrondis et des couleurs
\usepackage{pgfornament} %pour pouvoir ajouter des ornements
\usepackage{ragged2e} %gestion des alignements
\usepackage{soul}

%Fontes
\usepackage{aurical}
\usepackage[light,condensed,math]{iwona}
\usepackage{antpolt} 


					
																		%%%%%%%% Mes commandes/instructions			
		%%%%%%%% Mes couleurs
\definecolor{DarkGray}{gray}{0.3}


		%%%%%%%% Package tcolorbox
\makeatletter
%Boite pour les rubriques a bords longs (pas utilis�)
\newtcbox{\encadrerond}{on line,arc=3pt,colback=black,colframe=black,boxrule=0pt,boxsep=0pt,left=6pt,right=6pt,top=4pt,bottom=4pt}
\makeatother


		%%%%%%%% Mes commandes
		
%Titre, sous-titres, rubrique, stamp, auteur et lieu
		%formater un titre
		\newcommand{\titre}[1]{\textbf{\Huge{{\fontfamily{qzc}\selectfont\begin{spacing}{0.7}#1\end{spacing}}}}\vspace*{5mm}}

		%formater l'auteur
		\newcommand{\auteur}[1]{\vspace*{-3mm}\begin{small}\begin{bf}#1\end{bf}\end{small}}
		
		%formater l'auteur dans un article � une ou deux colonnes
		\newcommand{\auteurunedeuxcolonne}[1]{\begin{small}\begin{bf}#1\end{bf}\end{small}}

		%formater le lieu
		\newcommand{\lieu}[1]{, \small{#1}\vspace*{-1.5ex}}

		%sous titres dans un article(titre de paragraphes)
		\newcommand{\soustitre}[1]{\pgfornament[height=1em, anchor=south]{1}\hspace{1mm}\textbf{\large{#1}}\\}

		%th�me/mot-cl� d'un article
		\newcommand{\stamp}[1]{\hfill\colorbox{black}{\color{White}\normalsize{\textbf{#1}}}\vspace{-1mm}}
		
		%th�me/mot-cl� d'un article avec des bords ronds
		\newcommand{\stamprond}[1]{\hfill\encadrerond{\textbf{{\color{white}{\normalsize #1}}}}}
		
		%rubrique (Edito, L'essentiel, Jeux)
		\newcommand{\rubrique}[1]{{\Huge{{\color{DarkGray}\vspace*{-0.2ex}\pgfornament[width = .7cm, height = .7em, ydelta = -.6ex]{11}\,{\Fontlukas{#1}}\,\pgfornament[width = .7cm, height = .7em, ydelta = -.6ex, symmetry=v]{11}}}}}
		
		%Carr� de fin d'article
		\newcommand{\finarticle}{{\color{DarkGray}\hspace{1em}\pgfornament[height=1em, anchor=south]{14}}}
		
		%Indication que l'article continue sur la page suivante
		\newcommand{\suite}{{\color{DarkGray}\hspace{1em}\textbf{ >\hspace{.3ex}>\hspace{.3ex}>}}}

%Commande et environnement article
		%environement pour des articles avec colonnes balanc�es
				%Utilisation: \begin{article}[nombre de colonnes]{Titre}{Auteur}{Lieu} Texte de l'article \end{article}
		\newenvironment{article}[4][3]
			{\titre{#2}\vspace*{-2mm}\auteur{#3}\lieu{#4}\begin{multicols}{#1}}{\end{multicols}}

		%environement pour cr�er des articles avec colonnes non balanc�es
				%Utilisation: \begin{article*}[nombre de colonnes]{Titre}{Auteur}{Lieu} Texte de l'article \end{article*}
		\newenvironment{article*}[4][3]
			{\titre{#2}\auteur{#3}\lieu{#4}\begin{multicols*}{#1}}{\end{multicols*}}
		
		%environnement pour des articles � deux colonnes avec une meilleure gestion de l'espace entre le titre et l'auteur		
		\newenvironment{articlebis}[4][3]
			{\titre{#2}\auteurunedeuxcolonne{#3}\lieu{#4}\begin{multicols}{#1}}{\end{multicols}}	
		
		%environnement pour des articles � deux colonnes avec une meilleure gestion de l'espace entre le titre et l'auteur		
		\newenvironment{articlebis*}[4][3]
			{\titre{#2}\auteurunedeuxcolonne{#3}\lieu{#4}\begin{multicols*}{#1}}{\end{multicols*}}	
		
		%cr�er des articles � une seule colonne
				%Utilisation: \articleunecolonne{Titre}{Auteur}{Lieu} Texte de l'article
		\newcommand{\articleunecolonne}[3]
			{\begin{spacing}{1.2}\titre{#1}\end{spacing}\auteurunedeuxcolonne{#2}\lieu{#3}\vspace*{0.4ex}}
			
		%afficher deux articles, sur deux puis une colonne
			%Utilisation: \doublearticledeuxun{Texte1}{Texte2}
		\newcommand{\doublearticledeuxun}[2]
			{\begin{minipage}[t]{0.6466\linewidth}\setlength{\parskip}{0.8ex}#1\end{minipage}\hspace{0.02\linewidth}\vrule\hspace{0.02\linewidth}\begin{minipage}[t]{0.3133\linewidth}\setlength{\parskip}{0.8ex}#2\end{minipage}\vspace{4mm}}

		%afficher deux articles,sur une puis deux colonnes
			%Utilisation: \doublearticleundeux{Texte1}{Texte2}
		\newcommand{\doublearticleundeux}[2]
			{\noindent\begin{minipage}[t]{0.3133\linewidth}\setlength{\parskip}{0.8ex}#1\end{minipage}\hspace{0.02\linewidth}\vrule\hspace{0.02\linewidth}\begin{minipage}[t]{0.6466\linewidth}\setlength{\parskip}{0.8ex}#2\end{minipage}\vspace{4mm}}

%Initial capitals et lignes

		%initials capitals et small caps
		\newcommand{\icsc}[2]{\lettrine{#1}{#2}} 

		%lignes horizontales de s�paration entre les articles
		\newcommand{\ligne}{\vspace*{-1.4cm}\rule{\linewidth}{.6mm}\vspace*{-2mm}}

%Encadr�s et pub
		%encadr�s de plus d'une colonne
			% Utilisation: \encadrecolonnes{noDeColonnes}{Titre de l'encadre}{texte de l'encadre (supporte les commandes)}
		\newcommand{\encadrecolonnes}[4][0.968]
			{\fbox{\parbox{#1\linewidth}{\setlength{\parskip}{0.8ex}\vspace*{1ex}\titre{#3}\vspace{-1ex}\begin{multicols}{#2}#4\end{multicols}}}}

		%encadr�s d'une seule colonne
			%Utilisation: \encadre{Titre de l'encadre}{Texte de l'encadre (supporte les commandes)}
		\newcommand{\encadre}[3][0.90]
			{\vspace{2mm}\noindent\fbox{\parbox{#1\linewidth}{\setlength{\parskip}{0.8ex}\vspace*{1ex}\begin{spacing}{1.30}\titre{#2}\end{spacing}\vspace*{1ex}#3}}}
	
		%petits �critaux de pub
		\newcommand{\petitepub}[3]{\vspace{3mm}\\{\color{Gray}\rule{\linewidth}{1.5mm}}\vspace{1mm}\\\begin{minipage}{1.2cm}\includegraphics[width=1.2cm, height=1.2cm]{#1}\end{minipage}\hspace{2mm}\begin{minipage}{4.5cm}\textbf{#2}#3\end{minipage}\vspace{1mm}\hrule}
		%\newcommand{\petitepub}[3]{\vspace{3mm}{\color{Gray}\rule{\linewidth}{1.5mm}}\\[1mm]\begin{minipage}{1.2cm}\includegraphics[width=1.2cm, height=1.2cm]{#1}\end{minipage}\hspace{2mm}\begin{minipage}{4.5cm}\textbf{#2}#3\end{minipage}\nopagebreak\vspace{1mm}\hrule}

%Mises en exergue
		%mises en exergue au d�but d'un article
		\newcommand{\exerguedebutarticle}[2][]{{\fontfamily{phv}\selectfont\textbf{{\color{DarkGray}\textsc{#1}}}}\vspace*{-2mm}\\{\color{Gray}\rule{\linewidth}{.6mm}}\\\vspace*{-1mm}{\fontfamily{iwona}\selectfont\large{\textbf{{\color{Gray}\pgfornament[anchor=south, width=4mm]{18} }\vspace*{1mm}#2}}}\vspace*{1mm}\vspace*{-2mm}\\{\color{Gray}\rule{\linewidth}{.6mm}}}
		
		%mise en exergue au d�but d'un article � deux colonnes
		\newcommand{\exerguedebutarticledeuxcolonnes}[2][]{{\fontfamily{phv}\selectfont\textbf{{\color{DarkGray}\textsc{#1}}}}\vspace*{-2mm}\\{\color{Gray}\rule{\linewidth}{.6mm}}\\\vspace*{1mm}{\fontfamily{iwona}\selectfont\large{\textbf{{\color{Gray}\pgfornament[anchor=south, width=4mm]{18} }\vspace*{-1mm}#2}}}\\{\color{Gray}\rule{\linewidth}{.6mm}}}

		%exergue de d�but d'article � une colonne
		\newcommand{\exerguedebutarticleunecolonne}[2][]{\vspace*{2.5ex}\\{\fontfamily{phv}\selectfont\textbf{{\color{DarkGray}\textsc{#1}}}}\vspace*{-2mm}\\{\color{Gray}\rule{\linewidth}{.6mm}}\\\vspace*{-2ex}\begin{flushleft}{\fontfamily{iwona}\selectfont\large{\textbf{{\color{Gray}\pgfornament[width=4mm]{18} }\vspace*{1mm}#2}}}\end{flushleft}\vspace*{-1mm}\\{\color{Gray}\rule{\linewidth}{.6mm}}\vspace*{1ex}}

		%nouveau paragraphe dans l'environnement exerguedebutarticle
		\newcommand{\newpar}{\\[0.8ex]{\color{Gray}\raisebox{1.5mm}{\pgfornament[width=4mm]{18}} }}

		%mise en exergue au milieu des articles
		\newcommand{\exerguemilieuarticle}[1]{{\color{Gray}\rule{\linewidth}{.6mm}}\\[1mm]\textsc{{\raggedright{}{\fontfamily{phv}\selectfont\large{#1}}}}\\[-1mm]{\color{Gray}\rule{\linewidth}{.6mm}}\vspace*{1ex}}

%R�sum�s d'articles
		%r�sum� d'articles � mettre sur la page de garde pour attirer les lecteurs
		\newenvironment{essentiel}[1][2]{\rubrique{L'essentiel...}\vspace*{-2ex}\begin{multicols*}{#1}}{\end{multicols*}}

		%r�sum� de chaque article dans l'anvironement essentiel
		\newcommand{\resume}[4]{
		\pgfornament[width=\linewidth]{88}\\
		\begin{center}\vspace*{-8mm}{\large\textbf{#2}}\\
		\vspace*{-2mm}\pgfornament[width=\linewidth]{88}
		\end{center}
		\begin{flushleft}\vspace*{-4mm}\small{\color{DarkGray}#1}\\
		\normalsize{#3}\hfill{\small\textbf{Page #4}}\end{flushleft}
		\begin{center}
		\vspace*{-.4cm}
		\pgfornament[width=.8\linewidth]{75}
		%\pgfornament[height=5mm]{1}
		\end{center}}

%Jeux
		%d�finitions du mot crois�
		\newcommand{\maDef}[2]{\printDef{#1}{\hspace{-2ex}\textbf{.} #2}}

%Soulignement et url
		%url
		\newcommand{\myurl}[1]{{\color{blue}\uline{\url{#1}}}}
		
		%souligner avec des points
		\newcommand{\udensdot}[1]{\tikz[baseline=(todotted.base)]{\node[inner sep=1pt,outer sep=0pt] (todotted) {#1};\draw[densely dotted] (todotted.south west) -- (todotted.south east);}}

		%souligner avec des traits till�s
		\newcommand{\udensdash}[1]{\tikz[baseline=(todotted.base)]{\node[inner sep=1pt,outer sep=0pt] (todotted) {#1};\draw[densely dashed] (todotted.south west) -- (todotted.south east);}}


		%%%%%%%% Modification des sudokus
\renewcommand{\SudokuLinethickness}{1pt} %Change la largeur de la ligne des sudokus


		%%%%%%%% Modification des encadr�s
\setlength{\fboxsep}{1.5ex} %augmente l'espace entre le texte et les encadr�s


		%%%%%%% Package ulem(soulignage)
\renewcommand{\ULdepth}{2pt} %Soulignage pour les url (la ligne est plus proche du texte que pour un soulignage normal)


		%%%%%%%% Package url
\urlstyle{sf} %tt, rm, sf, same


		%%%%%%%% Package Lettrines
\setlength{\DefaultNindent}{0.5mm}
\setcounter{DefaultLines}{2}
\renewcommand{\DefaultLraise}{0} %Elevation de la lettrine par rapport � la basline
\renewcommand{\DefaultLoversize}{0.3} %facteur de grossissement de la lettrine vers le haut
			%Les deux commandes suivantes sont � utiliser si on veut utiliser une police sp�ciale pour les lettrines
			%\newcommand*\initfamily{\usefont{U}{Konanur}{xl}{n}}
			%\renewcommand{\LettrineFontHook}{\initfamily}


		%%%%%%%% Colonnes
\setlength{\columnsep}{0.5cm} %longueur de l'espacement entre les colonnes


		%%%%%%%% Les longueurs du document
%les longueurs de la feuille
\setlength{\topmargin}{-.54cm} %marge � ajouter en haut en plus de l'inch deja present
\setlength{\oddsidemargin}{-.1cm} %distance entre le texte et l'inch de marge du bord des pages impaires
\setlength{\evensidemargin}{-.1cm} %distance entre le texte et l'inch de marge du bord des pages paires
\setlength{\headsep}{.7cm} %Distance entre le header et le texte
\setlength{\headheight}{13.6pt} %hauteur du header
\setlength{\marginparsep}{0mm}
\setlength{\marginparwidth}{0mm}
\setlength{\textwidth}{16.56cm}
\setlength{\textheight}{23.7cm}
\setlength{\footskip}{1.3cm}

% Les longueurs des paragraphes
\setlength{\parskip}{1em} %espace en dessus du paragraphe (entre paragraphes)
\setlength{\parindent}{0pt}  %la longeur de l'alin�a au d�but des paragraphes



		%%%%%%%% Le header
\pagestyle{fancy}
 \lhead{{\bf Le Parchemin}}
	\cfoot{\thepage} %enlever le num�ro de page qui est au milieu du footer

 \rhead{décembre 2014}
\usepackage{rotating} 
\usepackage{calc}

\renewcommand*{\hrulefill}[1][0.3mm]{\leavevmode \leaders \hrule height #1 \hfill \kern 0pt}

\begin{document}
\thispagestyle{empty}
\begin{center}
\vspace*{-4.2cm}\hspace*{-5mm}\includegraphics[width=\textwidth+2cm]{../../images/logoParchEpureNoel3.png}
\begin{flushright}{\fontfamily{phv}\selectfont\vspace*{-1cm}décembre 2014}\hspace*{4mm}\end{flushright}
\vspace*{-10mm}\pgfornament[width = \textwidth]{86}
\vspace*{1mm}
\end{center}
\begin{center}
\begin{minipage}[c]{\textwidth}
\vspace*{-1.4cm}\includegraphics[width=\textwidth]{images/couverture2.jpg}
\end{minipage}\hspace*{0.7mm}
\end{center}
\vspace*{-.9cm}
\begin{center}
\hrulefill[.7mm]\hspace*{2mm}{\Huge{{\color{DarkGray}\vspace*{-0.2ex}\pgfornament[height = 1.5em, ydelta = -.6ex]{79}\,{\Fontlukas{L'édito}}\,\pgfornament[height = 1.5em, ydelta = -.6ex, symmetry=v]{79}}}}\hspace*{2mm}\hrulefill[.7mm]
\end{center}
\vspace*{-4mm}\begin{multicols}{3}
\vspace*{-.9cm}\auteur{Alicia Frésard}\\[.4mm]
Un peu de chaleur dans ce froid d’hiver...\\
La rédaction du Parchemin, après trois numéros à grand succès, se permet de vous
proposer un numéro tout en légèreté, propice à la décontraction post-trims. Parce que le mois de décembre est à la fois annonciateur de fin et de renouveau, il nous a semblé particulièrement à propos d’offrir à nos chaleureux et fidèles lecteurs un petit aperçu personnel de cette époque particulière. Crise entre les examens ? Passe-temps du dimanche ? Lecture de toilettes ? No soucis, le Parchemin vous accompagne en toutes situations, partout.\\
Mais ce mois-ci, nous vous souhaitons par-dessus tout un merveilleux Noël, de bons examens et une chouette nouvelle année 2015 !\quad\pgfornament[height=1em, anchor=south]{1}
\end{multicols}\vspace*{-1cm}\hrulefill[.7mm]

\newpage
\vspace*{-2.8cm}
\begin{spacing}{.94}
\begin{article}{Comment bien réussir ses trim's}
{Aline Mo \& Matthis E. Pasche}{Entre deux \st{incantations} révisions}
\exerguedebutarticle[Trucs et astuces]{Pour éviter la crise d'angoisse à ses lecteurs, le Parchemin dévoile ses meilleures techniques de révision.}
Cher collégien, ça y est, le cauchemar a commencé : les semestrielles sont là. Bien sûr, cela signifie également que les cours sont terminés pour 2014 (hourra !). Mais nous te connaissons bien, ami lecteur, et savons parfaitement que Flemme et Paresse sont tes principaux prédateurs, sans parler de la Fièvre du samedi soir (ou mercredi, ou lundi, ou jeudi, ou …). Et dans tout cela, que deviennent tes révisions? (Oups !)
Pas de panique ! Comme toujours, le Parchemin vole à ta rescousse et t'offre ses meilleurs conseils pour des semestrielles passées haut la main (et les notes) !

\soustitre{1. Stimule ton appareil cérébral}
Lorsque tu sens poindre la fatigue et approcher le moment où tu ne seras bon qu'à fixer les fissures du plafond, un filet de bave pendant de ta bouche entrouverte, n'hésite pas, réagis ! Tes neurones somnolent, il faut les réveiller. Pour cela, rien de tel que deux ou trois colonnes droites contre le mur de ta chambre. Elles activeront ta circulation sanguine et amèneront de l'oxygène à tes petites cellules grises !
Dans le cas où tu prendrais conscience trop tard de ton engourdissement et ne parviendrais pas même à te lever pour effectuer lesdites colonnes droites, assène-toi une bonne claque, ou verse-toi un verre d'eau froide dessus, tu seras immédiatement plus alerte !

\soustitre{2. Ne laisse rien ni personne te distraire}
Pour tes révisions, choisis un endroit calme et assez spacieux. Munis-toi de tous les documents nécessaires, puis trace un cercle autour de ton lieu de travail avec du gros sel. Cela empêchera les esprits malveillants de pénétrer ton espace de travail et te garantira une paix absolue. Petit truc : plus ton cercle sera régulier, plus la protection sera efficace !
Si tu es victime d'une personne particulièrement collante, lave-toi les cheveux avec un mélange d'huile de friture et d'œuf battu. Tes ennemis même les plus coriaces succomberont instantanément sous l'effet de cette recette miracle !

\soustitre{3. Mets la chance de ton côté}
Bois une infusion de trèfle à quatre feuilles le soir, si tu le fais le matin, il y a plus de risques que tu doives aller au petit coin en plein milieu de ton examen écrit… ; croise les doigts, mais pas lorsque tu écris, ou ton enseignant se verra obligé de te taxer d'un zéro pointé pour cause d'illisibilité ; tiens-toi loin des échelles, en période d'Escalade, le danger est partout ; et n'hésite pas à te relever la nuit pour guetter les étoiles filantes !

Enfin, cher lecteur, n'oublie pas que si tes efforts demeurent vains, Charlie est là pour faire disparaître les mauvaises notes…
\end{article}
\ligne
\begin{article}{Ze Kweshtion Thoux}
{Renat Arjantsev}{sur les boules}
%\exerguedebutarticle[Sous le sapin]{Le retour de Ze Kweshtion avec les cadeaux de Noël !}
Voici l'hiver qui s'approche à grands pas, augmentant le retard des trams et diminuant le mercure. Cette fin d’année rime aussi avec fêtes, cadeaux, retrouvailles en famille et délices culinaires. Personne n'est insensible au charme de la tradition des cadeaux sous le roi de la forêt et c'est justement de ces emballages bariolés que nous allons parler.
De toutes tailles et de toutes formes, de toutes les couleurs et de tous les gabarits, les cadeaux vont fleurir comme par magie au pied du roi de la forêt. Achetés par un oncle, offerts par une mère, ils vont ainsi ravir les cœurs des petits et des grands et sont attendus comme la dinde fourrée et le Père Noël. Parfois déçus, parfois heureux de ce que l’on a reçu, on est toujours content de partager ce moment magique avec nos proches et les personnes qui nous sont chères.
\exerguemilieuarticle{Vous n’avez pas idée de tout ce que l’on peut faire avec un trombone... ;)}
Même si on n’est pas supposé \enquote{commander} pour Noël, on a tous envie d’un petit quelque chose particulier : une bricole, une babiole, une déco sympa, des billets pour un concert, un spectacle. Noël, c’est un peu comme un deuxième anniversaire, sans le gâteau, mais avec plus de nourriture !

\soustitre{Et toi, tu veux quoi pour Noël ?}
Après avoir reçu vos réponses, on a fait un tri et voici quelques perles : certaines personnes ont avoué avoir envie de recevoir un animal (Quoi, tes frères et sœurs ne te suffisent pas ?), d’autres préfèrent ne rien espérer pour mieux être étonnés. Il y a tout de même quelqu’un qui veut devenir propriétaire d’un koala femelle et une certaine personne qui m'a avoué désirer plus que tout d’enfin réussir à voir le Père Noël “en vrai” et que pour ce faire il/elle allait ajouter des somnifères au lait qu'il lui laisse d’habitude.

\soustitre{Et toi, tu offres quoi à Noël ?}
Noël ce n'est pas seulement recevoir, mais c'est aussi et surtout offrir. La générosité de certains n'a pas de limite, car ils vont même jusqu'à offrir une robe de chambre à leur animal de compagnie avec son nom brodé dessus. (Si si, c'est vrai, je vous assure !)
Le plus dur à mon sens est de trouver le "bon" cadeau, celui qui fera plaisir et qui restera en mémoire, qui correspond le plus à la personne qui le reçoit et qui est aussi original. Les conseils que j’ai pu piocher parmi vous sont les suivants : du papier toilette, car on en a toujours besoin; une pierre : simple, rustique et pas cher. Le meilleur pour finir : un trombone... Pourquoi ? Vous n’avez pas idée de tout ce que l’on peut faire avec un trombone ;)
C’est sur cette dernière note d’humour que je vous laisse, je vous souhaite de bonnes semestrielles, de joyeuses fêtes de fin d’année et à l'année prochaine !
\end{article}
\ligne
\begin{article}{L'épopée de Khåňąbdøüd}
{Poub' et Anton Nomase}{dans une boule à neige}
\exerguedebutarticle[Nouvelle fantastique]{Préparez-vous à un combat dantesque entre un héros aux mille ruses et le plus grand, le plus vil, le plus méchant, méchant depuis Cruella d’Enfer.}
De petits flocons doux comme des plumes et légers comme du satin finissaient de voltiger sur la plaine apaisée. Elle s’était enfin calmée après sept affreux jours de tempête. Toutes les âmes de la petite bourgade étaient encore dans les bras du tendre Morphée. Toutes? Non. Le vaillant Khåňąbdøüd, la pupille frétillante et l'iris animé d'une flamme aventureuse, s'élançait à travers la toundra gelée, chevauchant son fidèle destrier: Beauregard, l'élan blanc.
Il voulait protéger sa mère patrie face au terrible danger qui la menaçait: l'Armadillo du désert et ses sbires, les tapirs. Après avoir conquis toutes les régions alentour, ils pensaient à présent pouvoir s'emparer sans peine du petit village de Netzenka. C'était sans compter sur Khåňąbdøüd et son amour pour la neige. Non, il ne laisserait pas l'Armadillo du désert aspirer toute la glace de la toundra. Non, il n'accepterait pas d'être spolié de son patrimoine ! Non, il ne permettrait pas à ces barbares de transformer le monde en un grand désert! Il allait se battre !

Surpris dans sa course effrénée par le guet-apens de trois vils tapirs sous la coupe du terrible Armadillo du désert, Beauregard se prend les pieds, enfin les sabots, dans un fil d’airain tendu entre deux esperluettes. Déséquilibre. Cri. Chute. Notre vaillant héros est à terre. On le traîne jusqu'à la tente de l'horrible Armadillo du désert. Avant même que notre homme ait pu émettre le moindre son, l'Armadillo du désert s’écrie dans un ricanement des plus machiavéliques :

-- Khåňąbdøüd ! Comme on se retrouve, mon ennemi juré ! C'est mignon de vouloir sauver son pauvre village, mais rien, rien ne saura m'arrêter ! Pas même toi, ta fougue et ta cornemuse !

-- Armadillo du désert, aujourd'hui sonne la fin de ton règne ! Les fées nous ont bénis ! Et par le pouvoir que me confère la sorcellerie évocatrice des rituels mithraiens, je vais te détruire par le funeste requiem Gaëlique.

À cet instant, le vent fraîchit et la montagne devint violette. Un bruit sourd s’éleva d’un mont avoisinant. Peu à peu, ce murmure destructeur gagna toute la vallée. Les esprits de la montagne s’animaient ! Chaque colline, chaque arbre, chaque ruisseau entonnait la même mélodie. C’était comme si l’on était revenu à l’aube des temps, au commencement même de l’univers. Cette mélodie si particulière résonnait en tout un chacun. Certains racontent que des aurores boréales et des arcs-en-ciel seraient apparus dans le ciel, en même temps que satyres et licornes dans les bois. Toute cette énergie convergeait en un seul point: la cornemuse de  Khåňąbdøüd. Il sortit sa cornemuse de son escarcelle, la dirigea vers l'Armadillo du désert et POUF ! L’Armadillo du désert et son armée n’étaient plus.
Morale de l’histoire: Mieux vaut savoir manier une cornemuse avec doigté, qu’une armée mal avisée.
Deuxième morale de l’histoire: Ce sont toujours les gentils qui gagnent.
\end{article}
\ligne
\begin{article}{\textsc{L'Escalade}: récit.}
{Le musicien}{pas loin de la cathédrale}
Lorsque l'on évoque l'Escalade, vous, jeunes étudiant-e-s, pensez tout de suite à la fête du collège, au défilé estudiantin, puis à une \st{orgie festive} petite soirée entre amis. Mais détrompez-vous ! Il s'agit aussi d'un splendide week-end de commémoration en vieille ville, commençant le vendredi soir et finissant par le mémorable cortège le dimanche soir. Pendant tout le samedi et le dimanche, si vous vous baladez en vieille ville, vous tomberez sur des groupes en tout genre: piquiers, fifres et tambours, mousquetaires et autres surprises ! Le grand cortège du dimanche soir rassemble tout de même 800 costumés, ce qui en fait un des plus grands cortèges historiques d'Europe. Quelques événements ne sont possibles qu'en ce temps: le passage Monetier (entrée rue du Perron 19), la dégustation de sanglier rôti à la broche sur la promenade Saint-Antoine ou encore la traditionnelle dégustation de vin thé chaud. Profitez de cette pause entre vos révisions de semestrielles pour accomplir cet acte patriotique qu'est la commémoration de l'Escalade, populace estudiantine genevoise !
\end{article}
\end{spacing}

\newpage
\begin{center}
\vspace*{-1cm}
\rubrique{Jeux}
\end{center}
\vspace*{-1.1cm}\titre{Labyrinthe}
\vspace*{-1.8cm}\includegraphics[width=\textwidth, height = 8cm]{images/labynoel3.png}

\vspace*{-1.7cm}\titre{logimage}
\begin{center}
\vspace*{-.8cm}\begin{minipage}{.9\textwidth}
\includegraphics[width=\linewidth, height=14cm]{images/BouleDeNeige3.png}\\
Jeux par Matteo et Stelaire
\end{minipage}\hspace*{1mm}\begin{minipage}{.06\textwidth}
\begin{rotate}{270}
\hspace*{-5.6cm}Pour faire ce jeu, suivez l'exemple donné dans le petit carré en haut à gauche
\end{rotate}
\end{minipage}
\end{center}
\end{document}
