\nomargintoc%
\nomarginimagecaption%
\chapter{Introduction}
\label{sec:introduction}

\leavevmode\marginElement{The \enquote{Affaire Maudet}}%
\marginElement{\input{images/sectioning/horizontal/\arabic{image_sectioning_horizontal}/readme.tex}}%
August 30\textsuperscript{th}, 2018, Geneva, Switzerland, the public ministry announces that it wants to investigate an elected official of the executive branch for illegally accepting benefits from a sheik in Abu Dhabi.
This is only the opening of the so-called \enquote{Affaire Maudet}, which will see many developments, and still be instructed at the federal level four years later.
In the face of public criticism and pressure, the magistrate refuses to resign, even after multiple bodies such as the executive chamber of the government and his party ask for his resignation letter.
There is no legal basis to revoke an elected official during its mandate, and so the magistrate will keep his position for two more years.
During this time, the other members of the executive create a new department with almost no influence over the conduct of the state, specifically for the suspected official.
The department he was previously assigned to, the police and justice department, is reassigned to the other magistrates thus impeding the smooth working of the body.
A law making it possible to revoke a magistrate is proposed and accepted by the people of Geneva during the November 21\textsuperscript{st}, 2021 votation.

\leavevmode\marginElement{The Yellow Vests}Winter 2018, the French yellow vests, a movement composed mostly of the lower classes, manifests every Saturday against their government by blocking roads.
They want the new proposed tax on fuel dropped, ask for more transparency from the state, more accountability, and the instauration of the citizens' initiative referendum among other revendications.
The protest makes the divide and lack of trust between the people and the French government explicit.
The movement sparks violent demonstrations; the Champs-Élysées and the Arc de Triomphe are ransacked on December 1\textsuperscript{st}, 2018.

\leavevmode\marginElement{Attack on the Capitol}January 6\textsuperscript{th}, 2021, supporters of Donald Trump attack the capitol in Washington, D.~C. after repeated allegations by the former president that the election has been \enquote{stolen} by the Democrats.
Trust in an established institution like the American voting process eroded at a speed that surprised many.

\leavevmode\marginElement{\textit{Roe v. Wade}}June 24\textsuperscript{th}, 2022, the supreme court overturns \textit{Roe v. Wade}, and so suppresses the federal right to abortion in the United States of America.
The decision has since been dubbed \enquote{one of the most undemocratic decisions} of the court, as 61\% of the American people think that it should have remained part of the law \cite{smith_mockery_2022}.

\leavevmode\marginElement{Democratic Issues}Many well-established democracies seem to be facing democratic crises.
Most governments that exist today were created centuries ago, at times during which the technological limitations were more drastic than today.
The ensuing systems had to be designed around those limitations, and therefore provide fewer guarantees than what might be possible and expected today.

\leavevmode\marginElement{The Open Source}When exploring new frontiers of governance systems, the open source ecosystem is inspirational material, as it has been at the forefront of technology, and thus explored new ways to coordinate enabled by modern means.
One of the key properties of open source is the ability to \textit{fork} a project.
Forking is the process of branching off from a project to start your own using the same starting code.
Many projects try to avoid forks as they split the community and the developer time each project receives.
Coordination is generally a better survival strategy in the open source.
This creates incentives for communities to have a governance process satisfying enough.
Note that it is much easier to fork an open source project than to ignite a revolution and bootstrap a new state.
As such, governance systems need to be more optimal in the open source, and when they are not, experiences are conducted at a much faster rate; projects fork, try, rise, and fall over a few months, not a few centuries.
Additionally, there are many more open source projects than there are states, and so the combination of plurality, and fast evolution in this ecosystem implies that it was subjected to a much more strenuous Darwinian selection.

\leavevmode\marginElement{The Blockchain}Blockchain is another domain worthy of attention when it comes to governance systems.
Because blockchains are well suited to transferring value, they enabled a new generation of governance systems that used this primitive to align individual incentives with the greater good more efficiently, than what could be achieved in the open source.
Coordination systems on the blockchain have experimented with various forms and colors, using fungible and non-fungible tokens, having explicit voting procedures or not, and featuring various token distribution mechanisms.
They have been applied to various goals ranging from coordinating a newsletter (BanklessDAO), to managing communities planting trees in Brazil (Toucan Protocol), to deciding the value of obscure protocol parameters of decentralized exchanges (MakerDAO).

\leavevmode\marginElement{Blockchain Speed}Because funds have been injected so quickly into the blockchain ecosystem, hitting an all-time high total value locked of more than 240 billion dollars in December 2021 \cite{noauthor_defillama_nodate}, the ecosystem evolved at an unrivaled speed.
Take for example the \textit{loot} project, an NFT collection of text strings describing adventure game gears, e.g. \emph{Bone Wand} or \emph{Pain Glow Scimitar of Brilliance}.
Brainstormed over a weekend, launched with a single tweet, after a week, the loot NFTs were traded at a staggering minimum value of 15 ETH (\$59'600)...
On the blockchain, the unit to measure the evolution of projects is the week, sometimes it's days.

\leavevmode\marginElement{Blockchain Narratives}The narratives related to blockchain are another reason it is a worthy experimentation ground.
In November 2018, the blockchain was invented by the cypherpunk movement, which notoriously distrusts everybody, especially nation states and banks.
To make a monetary system work without a trusted third party, they came up with various game theoretical mechanisms that make it possible to trust the correctness of a ledger without having to trust any single entity: the blockchain.
This defined the core values of the movement: \emph{decentralization}, \emph{trustlessness}, and \emph{permissionlessness} which all have wide-reaching consequences when applied in the context of governance.
Trustlessness: the system will work as intended without having to trust anyone.
Remember the attack on the American Capitol and the \enquote{stolen} election?
Decentralization and permissionlessness: distribute power in a more egalitarian fashion, and avoid having a minority holding all the power.
Remember \emph{Roe v. Wade}, remember the yellow vests?
Transparency; when everything is open, cheating becomes much harder.
Remember the \enquote{Affaire Maudet}?

\leavevmode\marginElement{Improve Governance of Open Source using Blockchain}We propose in this work to take a new look at governance systems, and what they can bring to open source projects.
How can we use blockchain functionalities to create provably net positive primitives to coordinate open source?
Is it possible to get feedback from users or the community?
Can we ensure that the governance system is inclusive for new contributors?
Can we guarantee that the outcome of the system maximizes the benefit of the entire community, not just a few members?
This work will investigate the use of \emph{decreasing value tokens}, and the limits imposed by the required \emph{sybil resistance}.

\leavevmode\marginElement{Improve Open Source Security}Blockchain can unlock new possibilities in other domains as well.
Take security; a single piece of software generally relies on dozens of libraries.
This creates the software supply chain, and it has been hacked: instead of attacking a well-watched program, corrupt its undefended dependencies.
Some of the most notorious hacks targeted a very simple Javascript library that was left abandoned, but which was used by many blockchain wallets \cite{noauthor_how_nodate}.
Can we provide mechanisms to improve security, like better code reviews or stronger guarantees that useful projects\\are not abandoned?
This work proposes a \emph{voting workflow}, that penalizes adversarial actors.

\leavevmode\marginElement{Improve Monetary Incentives for Open Source}Today, some open source software is embedded in every digital component that exists.
Open source creates priceless value for humanity.
Yet there are almost no developers in the open source ecosystem that are paid for the work they do or the value they create.
This lack of alignment between value creation and value extraction dissuades people from contributing, this is a coordination issue.
Can a blockchain-based system provide means to compensate developers?
This work proposes a \emph{rewarding scheme} and a \emph{monetary distribution mechanism} that incentivizes people to contribute to open source and realigns value creation and value extraction.

\leavevmode\marginElement{Structure of this Work.}
In this work, we will first analyze governance systems, the open source, and blockchain movements to understand them better.
Then, we will propose some blockchain primitives, what we call a \emph{GitDAO}, to improve various aspects of open source and make it more trustless.
Finally, we discuss our attempts at implementing GitDAO.

