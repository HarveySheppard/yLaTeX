% !TeX spellcheck = en_US
%%%%%%%%%%%%%%%%%%%%%%%%%%%%%%%%%%%%%%
%		Basic configuration
%%%%%%%%%%%%%%%%%%%%%%%%%%%%%%%%%%%%%%
\documentclass[a4paper, 11pt, oneside, fleqn]{article}
\usepackage[no-math]{fontspec}

\usepackage{polyglossia}
\setdefaultlanguage{french}
\setotherlanguages{english}



%%%%%%%%%%%%%%%%%%%%%%%%%%%%%%%%%%%%%%
%		Various packages
%%%%%%%%%%%%%%%%%%%%%%%%%%%%%%%%%%%%%%
\usepackage{metalogo}
\usepackage{microtype}
\usepackage{graphicx}
\usepackage{wrapfig}
\usepackage[german=swiss]{csquotes}
\usepackage{calc}
\usepackage[usenames,dvipsnames,svgnames,table]{xcolor}
\usepackage{amsmath, amsfonts, amssymb}
\usepackage{appendix}
\usepackage{setspace}
\usepackage{listings}
\lstset{
	basicstyle=\ttfamily,
	columns=fullflexible,
	keepspaces=true,
}



%%%%%%%%%%%%%%%%%%%%%%%%%%%%%%%%%%%%%%
%		Colors
%%%%%%%%%%%%%%%%%%%%%%%%%%%%%%%%%%%%%%
\definecolor{mainColor}{RGB}{211, 47, 47}

\newcommand{\inColor}[1]{{\bfseries\color{mainColor}#1}}



%%%%%%%%%%%%%%%%%%%%%%%%%%%%%%%%%%%%%%
%		Layout
%%%%%%%%%%%%%%%%%%%%%%%%%%%%%%%%%%%%%%
\usepackage[
	xetex,
	a4paper,
	ignoreheadfoot,
	left=3.5cm,
	right=3.5cm,
	top=3.5cm,
	bottom=3.5cm,
	nohead,
	marginparwidth=0cm,
	marginparsep=0mm
]{geometry}
\setlength{\skip\footins}{1cm}
\setlength{\footnotesep}{2mm}
\setlength{\parskip}{1ex}
\setlength{\parindent}{0ex}



%%%%%%%%%%%%%%%%%%%%%%%%%%%%%%%%%%%%%%
%		Tables
%%%%%%%%%%%%%%%%%%%%%%%%%%%%%%%%%%%%%%
\usepackage{array}
\usepackage{tabu}
\usepackage{longtable}

\definecolor{tableLineOne}{RGB}{245, 245, 245}
\definecolor{tableLineTwo}{RGB}{224, 224, 224}



%%%%%%%%%%%%%%%%%%%%%%%%%%%%%%%%%%%%%%
%		Links
%%%%%%%%%%%%%%%%%%%%%%%%%%%%%%%%%%%%%%
\usepackage{hyperref}
\hypersetup{
	pdftoolbar=false,
	pdfmenubar=true,
	pdffitwindow=false,
	pdfborder={1 1 0},
	pdfcreator=LaTeX,
	colorlinks=true,
	linkcolor=blue,
	linktoc=all,
	urlcolor=blue,
	citecolor=blue,
	filecolor=blue
}



%%%%%%%%%%%%%%%%%%%%%%%%%%%%%%%%%%%%%%
%		Itemize and consort
%%%%%%%%%%%%%%%%%%%%%%%%%%%%%%%%%%%%%%
\def\labelitemi{---}
\usepackage{enumitem}
\setlist[itemize]{nosep}
\setlist[description]{nosep}
\setlist[enumerate]{nosep}



\newcommand{\myTitle}{\inColor{\fontsize{1.3cm}{1em}\selectfont yReport}}



\begin{document}
	\everyrow{\tabucline[.4mm  white]{}}
	\taburowcolors[2] 2{tableLineOne .. tableLineTwo}
	\tabulinesep = ^4mm_3mm
	

%%%%%%%%%%%%%%%%%%%%%%%%%%%%%%%%%%%%%%
%		Title
%%%%%%%%%%%%%%%%%%%%%%%%%%%%%%%%%%%%%%
	\begin{flushleft}
		\begin{minipage}{\widthof{\myTitle}}
			{\fontsize{.6cm}{1em}\selectfont\color{mainColor}
				Documentation
			}
			\begin{spacing}{3}
				\myTitle
			\end{spacing}
			\vspace*{-10mm}
			\begin{flushright}
				Yves Zumbach
			\end{flushright}
		\end{minipage}
	\end{flushleft}
	
	
	\newpage
	{
		\hypersetup{linkcolor=black}
		\tableofcontents
	}
	\newpage
	
	\section{Prerequisites}
	You must compile this class with \XeLaTeX for it to work properly. You need to install the fonts that are in the \lstinline[breaklines]|fonts/| directory and finally to install the infoBulle package.
	
	\section{Class options}
	\begin{itemize}
		\item noColorBullet turns off the bullet coloration
		\item frenchBullet(default) turns on French typography for bullets (noFrenchBullet makes the opposite)
		\item french change the document settings to be in French
		\item article makes the document lighter (remove chapter counting, etc.)
		\item noHeaders disable the headers
		\item portable (see \ref{sec:portable})
	\end{itemize}
	
	\section{Page Layout}
	\lstinline[breaklines]|\symmetricalPage| changes the margin so that the page is symmetrical (no more margin par, small margin left and right).
	
	\lstinline[breaklines]|\asymmetricalPage| does the opposite than \lstinline[breaklines]|\symmetricalPage| and restore the margin paragraph, the asymmetrical margins, etc.
	
	\section{Length Commands}
	\lstinline[breaklines]|\wholeMargin| is the addition of the length of the margin paragraph and the marginparsep.
	
	\lstinline[breaklines]|\wholeWidth| is the addition of \lstinline[breaklines]|\wholeMargin| and the text width.
	
	\lstinline|\bigVerticalLineWidth| is the length of the vertical line used for chapter and marginpar backgrounding.
	
	\section{Font Commands}
	Following commands change the current font to the one they describe:
	\begin{lstlisting}
\normalFont
\titleFont % for title page
\headingFont % section, subsection, subsubsection
\chapterNumberFont
\chapterFont
\serifFont
\sectionNumbers
	\end{lstlisting}
	
	\section{Colors}
	To change the document color, use following syntaxe: \lstinline[breaklines]|\definecolor{mainColor}{RGB}{<red>, <green>, <blue>}|
	
	Following colors might also be redefined as described above:
	\begin{lstlisting}
sectionNumbersColor
subsectionNumbersColor
lightGrey
middleGrey
darkGrey
	\end{lstlisting}
	
	\lstinline[breaklines]|\isOddPage{<true>}{<false>}| is a command that check if a page is odd, execute <true> if it is the case, <false> otherwise.
	
	\section{Backgrounding Margin Paragraph}
	To add a background color to the margin par, you can use \lstinline[breaklines]|\backgroundThisPageGrey| or \lstinline[breaklines]|\backgroundThisPageColor|. I recommend using the first one as it is more discrete.	
	
	\section{Titlepages}
	I defined different titlepage format. Use:
	\begin{lstlisting}
\subtitle{<subtitle>}
\title{<title>}
\author{<author>}
...
\begin{document}
\titleOne
or
\titleTwo[front/image/path.jpg][<image_dimen>]
...
	\end{lstlisting}
	
	Personally, I prefer \lstinline[breaklines]|\titleTwo|. Its two optional arguments are used as follows \lstinline[breaklines]|\includegraphics[<image_dimen>]{front/image/path.jpg}|. Note that \lstinline[breaklines]|\titleTwo| supports up to three lines title. More lines will overflow from the page. You can use \lstinline[breaklines]|\\| in the \lstinline[breaklines]|\title| command.
	
	\section{Lists}
	Following environments should be used instead of the normal ones:
	\begin{lstlisting}
items (equivalent of itemize)
enum (equivalent of enumerate)
descr (equivalent to description)
	\end{lstlisting}
	
	Inside \lstinline[breaklines]|descr|, please, use \lstinline[breaklines]|\itemColor{<item>}| instead of \lstinline[breaklines]|\item|.
	
	Note that these environments also exists for margin par: \lstinline[breaklines]|sideItems, sideEnum, sideDescr|.
	
	\section{Margin Notes and margin title}
	To add some note in the margin, use the \lstinline[breaklines]|\sideNote{<note text>}| command. To make a margin title, use \lstinline[breaklines]|\sideTitle{title}|. Those title are just bigger. They are not numerated nor do they appear in the table of content.
	
	
	\section{Figures}
	\subsection{Body Figures}
	All figure should be created using \lstinline[breaklines]|\begin{SCfigure}[][ht!]| (as a replacement of \lstinline[breaklines]|\begin{figure}|). The first argument should be left empty, the second is the float placement.
	
	\subsection{Side Figure}
	\lstinline[breaklines]|\sideFigure[<caption>]{<figure>}| draw a table in the margin par. If you want the figure to spread to the margin paragraph width, use \lstinline[breaklines]|\includegraphics[width=\marginparwidth]{image/path.jpg}| when you include the figure.
	
	
	\section{Tables}
	\subsection{Body Table}
	For body tables, use \lstinline[breaklines]|tabu| commands inside a \lstinline[breaklines]|SCtable| environment:
	\begin{lstlisting}
\begin{SCtable}[][ht!]
	\begin{tabu}{<cols>}
		\tableHeaderStyle
		first & line & of & the table\\
		other & lines & of & the & table\\
	\end{tabu}
	\caption{<caption text>}
\end{SCtable}
	\end{lstlisting}
	If your table spreads across multiple pages, use the longtabu environment instead of tabu.
	
	For the table to spread to the text width: \lstinline[breaklines]|\begin{tabu} to \linewidth {<cols>}|. And then you will need to use X column in the table preamble (see tabu documentation).
	
	\subsection{Side Table}
	\lstinline[breaklines]|\sideTable[<caption>]{<table>}|
	
	If you want the table to spread to the margin paragraph width, use \lstinline[breaklines]|\begin{tabu} to \marginparwidth {<cols>}|.
	
	\section{Full Width Element}
	To make an element (generally a table or a figure) take the whole page (document body and margin paragraph), use the \lstinline[breaklines]|whole| environment. The content will be left or right aligned depending on the page being odd or even. If you want the content to be centered, use the \lstinline[breaklines]|centered| environment.
	
	Example:
	\begin{lstlisting}
\begin{figure}[ht!]
	\begin{whole}
		\includegraphics[width=\linewidth]{image/path.jpg}
		\caption{<caption>}
	\end{whole}
\end{figure}
	\end{lstlisting}

	\section{Quotation}
	For citations that fill the text width, use the \lstinline|\blockQuote[<author>]{<citation>}| command.
	
	For citation in the margin, use: \lstinline|\sideQuote[<author>]{<content>}|.
	
	Finally, for full width quotation, use \lstinline|\fullQuote[<author>]{<content}|.
	
	\section{Metadata}
	After calling the yReport class, please call the
	
	\begin{lstlisting}
\hypersetup{
	pdftitle={<Title>},
	pdfsubject={<Subject>},
	pdfauthor={<Your name>},
	pdfkeywords={{<keyword 1>}{<keyword 2>}},
}
	\end{lstlisting}
	macro and fill it accordingly to your document.
	
	\section{Headers}
	yReport defines some default headers. For them to work, please append the following code just before the \lstinline|\begin{document}| command:
	\begin{lstlisting}
\makeatletter
\let\runauthor\@author
\let\runtitle\@title
\makeatother
	\end{lstlisting}
	To disable the headers, pass the \lstinline|noHeaders| option to the yReport class.
	
	
	\section{Ornaments}
	The yReport class does provide four ornaments:
	
	\begin{lstlisting}
\ornamentOneTop
\ornamentOneBottom
\ornamentTwoTop
\ornamentTwoBottom
	\end{lstlisting}
	
	You can use them to mark the end of your chapter (although some say it does not match the overall design).

	\section{Side Math}	
	Math in margin par should be typeset without margin:
	\begin{lstlisting}
{\mathLeft
\[your math\]}
	\end{lstlisting}
	
\end{document}